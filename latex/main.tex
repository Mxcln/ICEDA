%!TeX program = xelatex
\documentclass[12pt,hyperref,a4paper,UTF8]{ctexart}
\usepackage{SHUreport}

%%-------------------------------正文开始---------------------------%%
\begin{document}

%%-----------------------封面--------------------%%
\cover

%%------------------摘要-------------%%
%\begin{abstract}
%
%在此填写摘要内容
%
%\end{abstract}

\thispagestyle{empty} % 首页不显示页码

%%--------------------------目录页------------------------%%
\newpage
\tableofcontents

%%------------------------正文页从这里开始-------------------%
\newpage

%%可选择这里也放一个标题
%\begin{center}
%    \title{ \Huge \textbf{{标题}}}
%\end{center}

\section{任务需求分析}   

\subsection{问题}

以上述IR作为输入,我们提供基本的IR的parser,根据上述IR,完成调度、寄存器及操作的绑定、控制逻辑综合,函数输入的数组综合为SRAM存储器,需要根据load/store指令来读写存储器数据。最终生成RTL代码。基本计算操作可以调用RTL计算模块,或直接使用原始操作符。

\subsection{需求分析}

1. 实现寄存器的绑定和调度
2. 实现控制逻辑综合

\section{团队分工情况}

\subsection{分工情况}

\begin{table}[h]
    \centering
    \begin{tabular}{|c|c|c|}
    \hline
    \textbf{姓名} & \textbf{分工} \\ \hline
    秦振航 & 寄存器的绑定和操作调度\\ \hline
    杨远达 & 控制逻辑综合 \\ \hline
    \end{tabular}
\end{table}

\section{程序实现思路与数据结构}

\subsection{代码结构}

项目的源代码在HLS/src文件夹中,包含以下模块:

\begin{enumerate}
    \item Paser 输入文件解析模块。
    \item main 主函数。
    \item Basic 基础类,包括语法块,句子,以及调度绑定时需要的点类。
    \item HLSHandler 高层次综合的实现类
\end{enumerate}

\subsection{命令行参数解析}

-i 输入解析文件名

\subsection{实现思路}

整体上分为两个步骤,首先将解析后的文件进行寄存器的绑定和操作调度,然后根据调度结果进行控制逻辑综合。

\subsection{寄存器的绑定和操作调度}

首先对于每个模块内,进行寄存器的绑定和操作调度。采用左边算法,

寄存器的绑定采用左边算法,

\begin{enumerate}
\item 如果是目标点,结束,返回。
\item 如果该点已经被锁定,已放入到了闭列表中,或者该点是障碍或者非法点,跳过该点。
\item 如果该点未被访问,计算他的代价,将父节点设置为取出的点,将他标记已访问之后放入开列表里。
\item 如果已在开列表中,且不是非法点,则重新计算其代价F,进行松弛,如果新计算的代价比原来的代价小,则更新开列表中对应点的代价和父节点,如果代价更大则不管。
\end{enumerate}

循环直到找到目标点。

\subsubsection{代价F的计算}

在寻径过程中,代价F由实际代价G和启发式估计H两部分组成。

实际代价G表示从起点到当前点的实际距离,在遍历节点的过程中,子节点的G为其父节点+1。

启发式估计H表示从当前点到终点的预估距离,在本题中,启发式估计H为当前点到中点的曼哈顿距离。

\section{测试方法和验证}

\subsection{测试方法}

在Astar文件路径下,执行命令行:

\begin{verbatim}
    .\build\bin\Astar.exe .\prob4.txt > .\output.txt
\end{verbatim}
    
也可以通过cmake自行编译。

\subsection{验证结果}

输入后的结果如下(见output.txt)

\begin{figure}[htbp]
    \centering
    \begin{minipage}[b]{0.45\textwidth}
        \centering
        \includegraphics[width=\textwidth]{figures/input.png}
        \caption{输入网格}
        \label{fig:image1}
    \end{minipage}
    \hfill
    \begin{minipage}[b]{0.45\textwidth}
        \centering
        \includegraphics[width=\textwidth]{figures/result.png}
        \caption{输出网格}
        \label{fig:image2}
    \end{minipage}
\end{figure}
\end{document}